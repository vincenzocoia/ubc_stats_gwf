\documentclass[11pt,a4paper]{article}
\usepackage{geometry}
\usepackage{url}

\title{Graduate writing workshop}
\date{\today}

\begin{document}

\maketitle

\section{Background}

Helping graduate students develop good writing skills is something the department has struggled for a long time like decades. This is also something other universities also struggle. 

The graduate writing workshop is organized one to two times per year.  The topic of a workshop should be related to writing using the "integrated" language of math and language. Examples of topics are math notation, writing abstracts, or technical report writing. In the workshop, students will be expected to exchange a small piece of their work with peers so that they can give and receive feedback using rubric provided by faculty. Faculty will also be invited to speak about the topic and participate in the student feedback discussion.  


\section{Tips for organizers}

The workshop is conventionally organized at the end of the term (either Term 1 or Term 2). The organizers should find a topic and a faculty speaker around two months in advance. See Section \ref{sec:topics} for topic suggestions and previous topics. The organizers may consult the department head for the choice of faculty that best matches with the chosen topic. Once the faculty speak is confirmed, the organizers should distribute an e-mail to graduate students to advertise the event. The e-mail should emphasize that students will be expected to exchange a small piece of their work with peers for feedback. The organizers should figure out the head-counts of participants at least two weeks before the workshop. 
Depending on the time of the workshop, the organizers may consider offering lunch or just snacks to the participants. See Section \ref{sec:catering} for catering suggestions. 
Here is the agenda the past organizers came up with. The total workshop length is 3 hours, and starts at 11 am. 

\begin{enumerate}
\item Introduction (discuss outline/agenda of the workshop (10 minutes). 
\item Presentation by a faculty member (30 minutes). Topics:
\begin{itemize}
\item Why is writing in the chosen topic important?
\item How can one write about the chosen topic well? Discuss and supply a rubric for ``evaluating" the topic of choice.
\end{itemize}
\item Feedback session (40 minutes). Note that students should enter the workshop with an example of their work on the topic of choice.
\begin{itemize}
\item Group students in pairs, e.g., students can work with the person next to them. 
\item Collect all the papers and randomly distribute them among the pairs. 
\item Each pair can discuss amongst themselves about how their peers' papers can be impoved (by using the rubric provided by faculty). Working in pairs encourages discussion, and is hence potentially more benecial than working as individuals.
\end{itemize}
\item Lunch (40 minutes). 
\item Individual discussion (30 minutes).
\begin{itemize}
\item Each pair would  explain their feedback to their peers, as well as receive an explanation of feedback on their own papers. (The organizers can decide the details of the mechanics.)
\item The faculty member could potentially be involved by
going from pair to pair and giving their input, or just be around to answer questions
that the pair may have.
\end{itemize}
\item Group discussion (30 minutes). Topics:
\begin{itemize}
\item What difficulties or controversies were there in giving feedback?
\item Provide one thing that you learn about how you could improve your writing
skills from the workshop. 
\end{itemize}
\end{enumerate}

\subsection{Topic suggestions}\label{sec:topics}

Here is the list of topics students are interseted of. When choosing a topic, the organizers should take the length of the work on the topic of choice into consideration. 

Previous topics were:

\begin{itemize}
\item The 2015 workshop: How to write an awesome abstract. (Faculty speaker: Paul Gustafson).
\item The 2014 workshop: Grant Application Writing. (Faculty speaker: John Petkau).
\end{itemize}

See Table \ref{tab:topics} for more suggested topics.

\begin{table}[thb]
\centering
\caption{Topic suggestions}\label{tab:topics}
\begin{tabular}{ll}
\hline
\# interests (/10) & Keywords \\
\hline
\multicolumn{2}{l}{\emph{Writing style and tone}}\\
5	& Writing abstract, introduction and conclusion\\
\multicolumn{2}{l}{Curriculum Vitae }\\
10	& CV \\
\multicolumn{2}{l}{\emph{Writing grant application}}\\
7	& Grant application \\
5	& Research summaries (past, current, and future) \\
4	& Technical (Stats jargons) vs. Non-technical language \\
\multicolumn{2}{l}{\emph{Documentation in programming}}\\
6	& Pseudo code\\
6	& Comment (programming)\\
6	& Naming convention (programming)\\
\multicolumn{2}{l}{\emph{Notation and mathematical conventions}}\\
7	& Notation and mathematical conventions\\
\hline
\end{tabular}
\end{table}


\subsection{Catering suggestions}\label{sec:catering}

The expected cost for organizing the event is conventionally about CAD180 for 15 people. The organizers may consult the head sec for catering suggestions. Potential funding sources include the GSS Event Fund, the department, and the Grad Fund. 
%Past organizers ordered some sandwiches and wrap from Wescadia. 


\end{document}
