\documentclass[11pt,a4paper]{article}
\usepackage{geometry}
\usepackage{url}

\title{UPDATES from 2014/2015 School Year}
\author{Organizers: Vincenzo Coia, Andy Leung, Neil Spencer}
\date{\today}

\begin{document}

\maketitle

\section{Overview}


The topic was ``How to Write an Awesome Abstract", and the faculty speaker was Paul Gustafson. The workshop was organized on April 30, 2015. Eleven students signed up for the workshop but nine students showed up (the numbers include the three organizers). The catering costed CAD182.75. 

The workshop started off with the speaker going over the rubric he chose for writing an abstract. Then the participants were grouped in pairs with the neighbouring person. The abstracts were collected and distributed among the pairs for review. Each pair had two to three abstracts to review. Lunch was given after the review session which was then followed by the feedback session. The participants were asked to look for one of the two reviewers of their abstract(s). Finally, the workshop was ended with a group discussion. In the group discussion, the participants were asked to share the difficulties or controversies they have encountered in giving feedback and to provide one thing they have learned from the workshop. 


\section{Ways to improve}

Overall, the workshop went very well, but there were some rooms for improvements.
\begin{itemize}
\item The number of attendees was less than the number of sign-ups. This wasted some money on catering. Future organizers could consider sending everyone sandwich/wrap choices and asking for individual orders. This way we can get a better estimate of the number of attendees, and we can also make sure everyone gets the sandwich/wrap they like.
\item The group discussion session was a bit too long. Future organizers could consider shortening it from 30 to 15 minutes. The organizers could just end the workshop with a self-reflection by asking the participants to think of one thing they have learned from the workshop. 
\end{itemize}

\section{Funding, now and future}

This year, we were able to get funding from the Grad Student Fund. The department paid for the catering first (so that there is no tax), then the organizers asked the treasurer to pay back the department. Future organizers should consider asking the department for funding, or applying for the GSS Event Fund as long as the organizers are part of the Statistics Graduate Students (a new Graduate Student
Organization (GSO) under the GSS).


\end{document}
